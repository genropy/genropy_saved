% ===== EXAMPLES ===========
%\chapter{...}  % only recognised for a book document class
%\section{...}
%\subsection{...}
%\subsubsection{...}
%\paragraph{...}
%\subparagraph{...}

% # $ % ˆ & _ { } ˜   % illegal characters

%\rule[lift]{width}{height}
%\rule{150mm}{.1pt}

% see section~\ref{sub:installing_python} on page~\pageref{sub:installing_python} 

% \begin{python}[moreemph={[4]42},caption={Simple python example No. 1},label=ex1]  % to show python code in latex
%\end{python}


%\begin{terminal}  % to create a terminal float
%  \begin{verbatim}
%\end{verbatim}
 % \caption{The Hello World! program in Java.}
%\end{terminal}


% ===== END EXAMPLES ===========

\begin{abstract}
Genro is simply a word, a word to describe a way of thinking, and applying that process of thought to the art of computer programming. Genro's author, Giovanni Porcari from Softwell, Milan Italy, is humble, unassuming and completely unpretentious, yet has an unrivalled passion and zeal to implement the Genro philosophy into all his work.  This document covers the work of Genropy, the Genro implementation of Python and dependencies to create an application development framework for building Web2.0 applications. The Genro philosophy includes the reduction of complexity to simplicity, and seeks to avail users of the framework with the ability to perform tasks in the easiest way possible, without sacrificing power or flexibility.

\paragraph{Motivation} % (fold)
\label{par:motivation}
\emph{Why do we care} about the problem and the results? If the problem isn't obviously ``interesting" it might be better to put motivation first; but if your work is incremental progress on a problem that is widely recognized as important, then it is probably better to put the problem statement first to indicate which piece of the larger problem you are breaking off to work on. This section should include the importance of your work, the difficulty of the area, and the impact it might have if successful.

% paragraph motivation (end)

\paragraph{Problem Statement} % (fold)
\label{par:problem_statement}
What \emph{problem} are you trying to solve? What is the scope of your work (a generalized approach, or for a specific situation)? Be careful not to use too much jargon. In some cases it is appropriate to put the problem statement before the motivation, but usually this only works if most readers already understand why the problem is important.

% paragraph problem_statement (end)

\paragraph{Approach} % (fold)
\label{par:approach}
\emph{How did you go about solving} or making progress on the problem? Did you use simulation, analytic models, prototype construction, or analysis of field data for an actual product? What was the extent of your work (did you look at one application program or a hundred programs in twenty different programming languages?) What important variables did you control, ignore, or measure?

% paragraph approach (end)

\paragraph{Results} % (fold)
\label{par:results}
What's the answer? Specifically, most good computer architecture papers conclude that something is so many percent faster, cheaper, smaller, or otherwise better than something else. Put the result there, in numbers. Avoid vague, hand-waving results such as ``very", ``small", or ``significant." If you must be vague, you are only given license to do so when you can talk about orders-of-magnitude improvement. There is a tension here in that you should not provide numbers that can be easily misinterpreted, but on the other hand you don't have room for all the caveats.

% paragraph results (end)

\paragraph{Conclusion} % (fold)
\label{par:conclusion}
\emph{What are the implications} of your answer? Is it going to change the world (unlikely), be a significant ``win", be a nice hack, or simply serve as a road sign indicating that this path is a waste of time (all of the previous results are useful). Are your results general, potentially generalizable, or specific to a particular case?

% paragraph conclusion (end)

\end{abstract}

\chapter{Introduction} % (fold)
\label{cha:introduction}

\section{Philosophy} % (fold)
\label{sec:philosophy}
A genropy core focus is on information meta data rather than a focus on methods or classes. This meta data describes information about data structures, web page structures, permissions, pdf reports, etc. instead of having this information frozen in classes. The meta data is stored in the database along with other data sets, or in xml files, or supported in python code as data.

 \emph{Gio, we need more for the philosophy of Genro}

% section philosophy (end)


\section{Things to Learn} % (fold)
\label{sec:Things_to _earn}
To understand how to build the Genropy Application Framework you should have an understanding of the Python programming language and be familiar with PostgreSQL, with a solid understanding of SQL. In addition it wouldn't hurt to know Javascript, at least enough to know the concepts, since you will be using the Dojo javascript libraries. You should read from the Dojo website and become familiar with the different widgets available and how they impact the interface. CSS classes are heavily used in Genropy so you will need to have a good understanding of how and where to use them. You'll also be downloading and configuring Apache 2 so you need to learn a little about how these configuration files are structured. In order for the webserver, python and postgres to talk to each other, you will need to download and install some dependencies. These are little support applications that help put the whole framework together. We will list all necessary items later on together along with some helper applications. Then you will need to learn the Genropy modules, classes, concepts and functions that create your applications quickly.

It may all sound too hard, but stay with us, as you will find that you will be building the very best applications in next to no time.  Genropy has so many things prebuilt that you will not have to reinvent the wheel to perform common tasks.
% section Things_to (end)

% chapter introduction (end)